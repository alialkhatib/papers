%!TEX root = ./main.tex
\documentclass[main]{subfiles}

\begin{document}

\section{Discussion}

\topic{Street--level bureaucracies are not a perfect metaphor for the phenomena we've discussed.}
We didn't address in any capacity the fact that
street--level bureaucrats sometimes diverge in unexpected ways from the prerogatives of their managers.
This becomes the source of tension in
\citeauthor{lipsky1983street}'s treatment of street--level bureaucracies,
but in our discussion of the relationships between street--level algorithms and their stakeholders,
we avoided the relationship between engineers and designers and the systems themselves.
Suffice it to say that while there is a disconnect between intent and outcome,
the nature of that relationship is so different that it warrants much further discussion.
We also avoided a unique quality of machines in this nuanced tension:
algorithmic systems operate at far greater speed than humans can~\cite{Matthias2004},
precipitating what \citeauthor{doi:10.1177/2053951716679679} characterized as a
``technical infeasibility of oversight''~\cite{doi:10.1177/2053951716679679}.


\topic{Nor are street--level bureaucrats paragons of
justice, fairness, accountability, transparency, or any particular virtue.} % \ali{this might be a little too ``do you get it yet???''}
Street--level bureaucrats have historically been agents of immense prejudice and discrimination:
writing insuring guidelines specifying that racially integrated neighborhoods are
inherently less safe than white ones~\cite{rothstein2017color}, for instance.
\citeauthor{whyte2012street}'s ethnography of organized crime in Boston,
and of a corrupt police force
that took payoffs to exercise their discretion more favorably toward criminal enterprises~\cite{whyte2012street},
illustrates in a different way how discretion can be applied contrary to our values.
Street--level bureaucracies are loci of immense power
--- and power can be abused by those who have it.
% The goal is not for algorithms to replicate the failures of bureaucrats.

\topic{Perhaps least certain of all the questions that emerge as a result of
this discussion of street--level algorithms is that of the relationship between conflicting agents.}
What happens when street--level bureaucrats collide with street--level algorithms?
The theory of street--level bureaucracies doesn't offer much to mitigate this tension.
\citeauthor{Veale:2018:FAD:3173574.3174014,doi:10.1177/2053951717718855} have
traced the landscape of challenges that may emerge
and ways to mitigate those conflicts~\cite{Veale:2018:FAD:3173574.3174014,doi:10.1177/2053951717718855}.
This area in particular needs further study:
the fault lines are here to stay, and we need to reflect on
this shifting of discretion from the bureaucrat to the engineer~\cite{eubanks2018automating}.
A value-sensitive approach~\cite{zhuvalue} would ensure that engineers be careful of
how the algorithms may support or undermine bureaucrats' authority.

\topic{Algorithms and machine learning may yet introduce new methods that override the observations made here.}
We are assuming, as is the case currently, that
algorithms require feedback or additional training to update their learned models.
Likewise, we are assuming that algorithms will continue to make errors of confidence estimation, and
will make mistakes by labeling marginal, novel cases with high confidence.
Nothing about the emerging architectures of modern machine learning techniques challenges these assumptions, but
should it happen, the situation might improve.

\topic{Despite these limitations, we suspect that
the lens of ``street--level algorithms'' gives us a starting point on many questions in HCI and computing more broadly.}
We've discussed the ways that
street--level bureaucracies can inform how we think about
YouTube content moderation, judicial bias, and crowdwork,
but we could take the same framing to task on a number of other cases:
\begin{itemize}[leftmargin=*]
  \item \textit{Moderation of forum content: } For many of the same reasons that
           we see trouble with the application of algorithmic classification systems on YouTube,
           we should expect to see problems applying algorithmic systems to textual forums.
  \item \textit{Self-driving cars: } Cars will make algorithmic decisions all the way
         from Level 1, where vehicles decide to break when we get too close to others,
           to Level 3, where they will need to decide when to hand control back to drivers,
           to Level 5, where systems may decide which routes to take
           and thus how late we should be to our destinations.
           Self-driving cars are literal ``street--level'' algorithms.
  \item \textit{Parental controls: } Algorithms that
          lock children out after a certain amount of screen time elapses will need to learn how to handle unforeseen situations
          when the device should remain functional, such as a threat or emergency.
  \item \textit{AI in medicine: } When decisions are life-or-death,
          how does a patient or doctor handle an algorithm's potentially error-prone recommendations?
\end{itemize}


\onlyinsubfile{
  \bibliographystyle{ACM-Reference-Format}
  \bibliography{references}
}

\end{document}

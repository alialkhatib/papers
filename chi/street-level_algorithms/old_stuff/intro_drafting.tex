\documentclass[main]{subfiles}

\begin{document}
\section{Introduction}


% \onlyinsubfile{

revising vs reinforcing?

algorithms don't \textbf{reflect} on the marginality of the cases they see.
it doesn't understand the concept of the margin

\begin{enumerate}
  \item {Digital systems mediate a lot of our lives}
  \item {Researchers have started thinking about these issues, developing a field of research that thinks more formally and critically about the place that algorithms take in our lives}
  \item {But we're still struggling to theorize about the roles these systems take in our lives}
  \item {bureaucratic theory talks about street--level bureaucracies, the layer of bureaucratic organizations where people make crucial decisions that define policies at the margins of legislated situations.}
  \item {These decisions they make become enormously consequential, because they ``fill in the gaps'' between prescribed behavior which can only be loosely prescribed in cases involving people.}
  \item {In this paper, we'll show that algorithmic systems are taking up the roles of street--level bureaucrats, becoming ``street--level algorithms'', and show how this way of thinking about these systems can help inform the design of these systems.}

  \par\noindent\rule{\textwidth}{0.5pt}

  \item \thesis{``Street--level algorithms'' make decisions about myriad cases just like street--level bureaucrats, but unlike bureaucrats their decisions don't continually inform the ongoing revision of policy that bureaucrats engage in.}
  \item we'll show how algorithmic systems ``fill in the gaps'' in marginal cases, sometimes classifying similar cases differently, but never using those cases, which we'll call ``salient cases'' for now, to prompt reflection on the decisions they make.
  \item We'll discuss some of the ways that designers of sociotechnical systems can mitigate some of the shortcomings in algorithmic systems today.
  \item This argument isn't gripping me. \ali{The argument that they should be reflecting on the marginality of the case and learning and changing over time would make more sense if this was a decades--long thing (maybe some really advanced AI that was trained in the 50s that didn't mature along with society, for instance). Instead what's happening is coming quickly. The problem seems to stem from not understanding the intuition that defines the nuance at these margins, and these systems aren't incentivized to make risky guesses. Does this just mean ``more data'' a sufficient answer to this problem? Because if so, ???}
  
  \par\noindent\rule{\textwidth}{0.5pt}\setcounter{enumi}{6}
  \item \thesis{``Street--level algorithms'' have become the focus of so much frustration in the public --- and the focus of so much attention and reasoning in the academic discourse --- because \textbf{they don't act as advocates for the people they serve}.} %, both in identifying whom to advocate for and how to advocate for them.}
  
  \par\noindent\rule{\textwidth}{0.5pt}\setcounter{enumi}{6}
  \item \thesis{``Street--level algorithms'' have proven so frustrating because, when they intuit appropriate decisions to ``fill in the gaps'', \textbf{they effect policy without reflecting on the consequences} of those decisions, and consequently \dots (???)}
  
  \par\noindent\rule{\textwidth}{0.5pt}\setcounter{enumi}{6}
  \item \thesis{``Street--level algorithms'' make the wrong decisions in marginal cases precisely because they don't understand their task, let alone underlying goals, and inevitably \textbf{they make decisions that advance their tasks, but not their goals}.}

  \par\noindent\rule{\textwidth}{0.5pt}\setcounter{enumi}{6}
  \item \thesis{``Street--level algorithms'' don't effectively parse out the difference between marginal and non--marginal cases. As a result, cases at the margins don't prompt SLAs to reflect on the decision space that they're operating in.}

  % \par\noindent\rule{\textwidth}{0.5pt}\setcounter{enumi}{6}
  % \item \thesis{``Street--level algorithms'' are doing great actually everything's fine relatively speaking keep up the good work}

\end{enumerate}
% }




\end{document}
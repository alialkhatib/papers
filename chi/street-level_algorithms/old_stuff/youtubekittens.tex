\documentclass[main]{subfiles}

\begin{document}


\onlyinsubfile{
\subsection{\ali{what would have happened if the system had some self--awareness (i.e. what if the AI were like \textit{actually} intelligent?)}}
}

\topic{An intelligent content classification system, independent of limitations involving computing capacity, would be able to fully appreciate the content as well as the context in which it was produced.}
In street--level bureaucracies, police officers take in the context of the performance before engaging (or choosing not to engage);
the time, place, the performer, the substance of the performance, and myriad other factors come into the decision in a split second.
It's more than the immediate differences between a musician playing in a subway and someone engaging in a Three Card Monte.
It's the reflection of whether each of these situations matters to the organization to which the street--level bureaucrat reports.
To a police officer,
it's clear that arresting the con artist serves the police department's goal, while
arresting the musician does not.
An algorithmic system attempting to replace a police officer in this scenario wouldn't be able to differentiate between benign street performers and malicious ones until it had been trained to understand characteristics that differentiate the two;
we're even further from an algorithm coming to appreciate what kinds of things its designers value more than others, except in a very formally described incentive system.

\onlyinsubfile{
\subsection{\ali{how do we make a stopgap to that end? (should be bureaucratically informed!)}}
}


\topic{Algorithmic systems may never be able to anticipate every instantiation of human creativity, especially in experimental performance art, which often aims to subvert expectations and surprise audiences.}
But it certainly is plausible to design systems that recognize when words and visual patterns begin to appear in unfamiliar contexts and ways.
When words appear to be used in novel contexts, or when imagery appears in unfamiliar places,
it can signal a new interpretation or meaning of those ideas or imagery, and prompt a reevaluation of how the system deals with them.\footnotemark{}
And in the cases where the classification system utterly fails to recognize the uniqueness of the material in question
(for instance, if some external context proves relevant),
representing the embedding the machine learning algorithm generated in classifying a person's work can help the person better understand the general cohort into which they've been grouped.
This can greatly inform a human--driven appeal process.

\footnotetext{Some examples that come to mind:
\begin{enumerate}
  \item people using sexual terms differently from sexually explicit (or otherwise previously intended) ways.
  \item when a community ``reclaims'' a term, changing the contexts in which it's acceptable (e.g. the `N' word).
  \item when a community reappropriates a term to use it pejoratively or for some other malicious purpose (e.g. ``I self--identify as an attack helicopter'').
\end{enumerate}}


\end{document}
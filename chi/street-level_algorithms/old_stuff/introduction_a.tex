% governance (is that how fb isn't really public?)
\documentclass[main]{subfiles}

\begin{document}

\section{Introduction a (governance and false publics)}\label{sec:introduction a}

\ali{The internet isn't public! or ``Why has democracy largely failed on the web?''}

{For the better part of 30 years, the ethos expressed by \citeauthor{barlow2009declaration} in \citeyear{barlow2009declaration}
has motivated how we talk about the web~\cite{barlow2009declaration}.}
Inspired by the Declaration of the Independence of the United States of America, \citeauthor{barlow2009declaration}
envisioned and advocated in cyberspace a world that was untangled from the limitations that physical embodiment carried,
freeing people to express whatever they want, believe whatever they want, and ultimately to \textit{be} whoever or whatever they want. \ali{even a dog! [insert dog new yorker internet comic here]}





% \topic{For decades, people have described the web as a place free of coercive forces we see offline --- a space disconnected from privilege or prejudice, one that transcends identity; a civilization free of the constraints of geographically situated governance~\cite{barlow2009declaration}.}
% In the years since \citeauthor{barlow2009declaration}'s declaration, laws have emerged regulating the behavior of myriad stakeholders on the internet \ali{DMCA, net neutrality, GDPR}, but \ali{largely we regard the web as an open, free space}.

{But it's not clear that the web has delivered on that promise in any sense.}
Legal regimes, such as copyright enforcement policies, net neutrality, the GDPR, and others effectively mediate cyberspace from physical space, assuredly to the dismay of people like \citeauthor{barlow2009declaration} and those who saw the potential in the future he articulated.

But more deeply, the web that \citeauthor{barlow2009declaration} described no longer exists, if it ever did.
In this paper, we take a tour of online platforms and services, illustrating a relationship between people and the designers of platforms that suggests that while we are indeed denizens of cyberspace, we are in fact serfs to feudal spheres of near--total hegemonic influence over our lives \ali{or something}.

\ali{Things we'll do:
  \begin{enumerate*}
    \item illustrate that people are effectively stuck in various platforms
    \item show that these platforms(/companies) make unilateral decisions, and only back down in the face of overwhelming pressure.
  \end{enumerate*}
}





% It's not free, or open, or democratic, in any of the senses that individuals broadly think about these concepts.
% Instead, we live in under numerous authoritarian regimes, mostly portraying themselves


\onlyinsubfile{
  \bibliographystyle{ACM-Reference-Format}
  \bibliography{references}
}
\end{document}

% non-human bureaucracies (implementation of policy/governance)
\documentclass[main]{subfiles}

\begin{document}

\section{Introduction b (algorithmic enforcement of rules on the web)}\label{sec:introduction b}

% The internet's spheres of influence are being and adjudicated algorithmically, and that's fundamentally not going to work.

\topic{Online systems scale effectively because we've delegated the mediation of much of our behavior to automatic systems.}
More than 300 hours of content gets uploaded to YouTube every minute;
\ali{lots of posts on reddit and twitter and facebook}.
And as we enumerate more policies, the nuance of those policies grows.
We used to enforce spam filters using pattern--matching strings catch certain key words or phrases;
today AI systems make these decisions, informed by dramatically more factors and making much more nuanced conclusions than the simple regular expressions that may have sufficed 20 or 30 years ago.

As algorithms for moderation get more advanced and indeed more nuanced, we assign them the task of adjudicating commensurately more nuanced problems.
Algorithmically mediated moderation systems determine the intent of editors on Wikipedia, the provenance and legal status of content that YouTubers produce, and whether certain people are eligible for certain work.
Unsurprisingly, the promotion of algorithms to these increasingly authoritative roles has transpired amid controversy:
\ali{myriad examples here}.

Researchers have looked into the behavior of users in defiance of (or sometimes oblivious to) the intent of the designers of systems.
Our paper hopes to show that these individual phenomena are not anomalous, unrelated points, but part of a holistic space that we can see with some clarity through a single theoretical lens --- that of bureaucratic theory as a mechanism to execute policy and governance, mediated by algorithms as \ali{autonomous bureaucrats bleh this is terrible}.
As we describe the relationships between the systems that mediate, and their stakeholders, through this lens, we hope to show that the ways people work \textit{around} sociotechnical systems can be made sense of \ali{ahhhh}
We also hope to bring into relief some of the ways that digitally mediated platforms \textit{diverge} from this framework, suggesting some useful ways to think about why tension arises with the users of these systems.

\ali{We'll do this by doing a couple of things}
\begin{enumerate}
  \item Review what bureaucratic theory tells us about how institutions work in the Weberian theoretical sense\ali{, acknowledging some of the problems that have emerged with the rise of neoliberalism but not getting bogged down in that because I really want to get on to the other stuff!}
  \item Describe how some online platforms work, using some of the vocabulary that bureaucratic theoretical literature affords us (YouTube, Uber, Facebook).
  \item Explore a few common problems that arise in the use of these platforms, highlighting the bureaucratic workarounds that are taking place, and what this might mean for designers of these and similar platforms.
\end{enumerate}

Ultimately, we hope that this framing will provide designers of systems with an additional process for thinking about sociotechnical systems at a more macro level.


what's the thesis here?

BT tells us that certain values are important in a bureaucracies



what we need in a beta paper
\begin{enumerate}
  \item thesis: what's the concrete takeaway? what are the values that people should think about when they design sociotechnical systems? what principle(s) define effective(?) bureaucracies in the senses that we're talking about? \ali{why are people upset at these systems? maybe: why not upset at other things?} \textbf{find one thing that has the outsize role in whether the bureaucratic system }

  it shouldn't be the same recommendations as FAT* (fairness, accountability, transparency)
  \item rest of the paper:
  \begin{enumerate}
    \item triangulate case studies; introduction of BT literature
    \item discussion bullet points? lol
  \end{enumerate}
\end{enumerate}

% This paper will introduce \textbf{bureaucratic theory} as a way of thinking about the relationships that transpire between people and the systems with which they interact online.


% In this paper, we attempt to offer a framework that describes these kinds of power negotiations between the designers of systems, the systems themselves, and the people upon whom these systems largely act.



\onlyinsubfile{
  \bibliographystyle{ACM-Reference-Format}
  \bibliography{references}
}
\end{document}

\documentclass[presentation]{subfiles}
\onlyinsubfile{
  \bibliography{references}
  \usepackage{tikz}
  \usetikzlibrary{positioning}
\renewcommand{\onlyinsubfile}[1]{}

}
% \setbeamersize{text margin left=0cm, text margin right=0cm}
% \setbeamersize{text margin left=0cm}

\begin{document}

\section{Everything old is new again}


\begin{frame}{on-demand work} % \frametitle{The beginnings of something frustrating}

    \visible<+->{
    \begin{columns}
      \begin{column}[t]{0.5\textwidth}
        \centering
        Gig work

        \vspace{0.5em}
        \end{column}
        \begin{column}{0.5\textwidth}
        \centering
        Crowd work
        \end{column}
      \end{columns}
      }

\vspace{2em}

\visible<+->{
  \begin{columns}[onlytextwidth]

    \begin{column}{.25\textwidth}
      \includegraphics[width=\textwidth]{../../common_figures/uber.png}
    \end{column}
    \begin{column}{.25\textwidth}
      \includegraphics[width=\textwidth]{../../common_figures/doordash.png}
    \end{column}

    \begin{column}{.25\textwidth}
      \includegraphics[width=\textwidth]{../../common_figures/amt.png}
    \end{column}
    \begin{column}{.25\textwidth}
      \includegraphics[width=\textwidth]{../../common_figures/upwork.png}
    \end{column}


    
  \end{columns}
  }

\end{frame}




% \begin{frame}<1>[standout,label=juxtapose]
%   \begin{columns}
%     \begin{column}{0.5\textwidth}
%       \begin{center}
%         {\large nothing new under the sun}
%       \end{center}
%     \end{column}
%     \begin{column}{0.5\textwidth}
%       \begin{center}
%         \onslide<1>{\large everything old is new again}
%       \end{center}
%     \end{column}
%   \end{columns}
% \end{frame}

\begin{frame}<-6>[t,label=pastProjects]
\frametitle{
  % \hspace{1em}
  \visible<1->{
    Gig Work
  }
  \only<4-9>{
    \hspace{10.5em}
    {Crowd Work}
  }
}
  
  \begin{columns}
  \visible<2->{
    \begin{column}[T]{0.5\textwidth}
      \begin{center}
        \textbf{\textsc{Alia}} % \\{\small(National Domestic Workers' Alliance)}}}
      \end{center}
    

      Allowing workers to
      book clients,
      communicate with each other, and
      make decisions about how the platform should operate

  \end{column}
}

  \visible<5-10>{
    \begin{column}[T]{0.5\textwidth}
    \only<-9>{
          \begin{center}
            \textbf{\textsc{Vitae}} % \\{\small(TurkerNation)}}}
          \end{center}
    
          Giving crowd workers the ability to stand behind collective reputations, and attract more skilled, lucrative work
        }
  \only<10-12>{
    \centering
    \includegraphics[width=0.8\textwidth]{figures/news/alia_portable_benefits_cropped.png}
  }
  \end{column}}
  \end{columns}


\begin{columns}

\begin{column}{0.5\textwidth}
  \begin{center}
    \only<3-12>{
      \visible<3-6,7-11>{
        \only<3-6>{
          \strong{Worker-owned cooperative}
        }

        \only<7->{
          \onslide<0>{
            Worker-owned cooperative
          }
          \visible<8->{

            \strong{Relief from precarity}
          }
        }
      }
      % \only<7-11>{
        
      % }
    }
  \end{center}
\end{column}

\begin{column}{0.5\textwidth}
  \begin{center}
    \visible<6-9>{
      \visible<6,7-10>{
        \only<6>{
          \strong{Digital hiring hall}
        }
        \only<7->{
          \onslide<0>{
            Digital hiring hall
          }
          \visible<8->{

            \strong{Solidarity}
          }
        }
      }
    }
  \end{center}
\end{column}

\end{columns}

\visible<9>{
  \vspace{1em}
  \begin{center}
    \strong{Solutions to underlying social issues}
  \end{center}
}
\end{frame}



\begin{frame}<0>[label=thesis] % {Underlying patterns}
  \begin{center}
    \alert<+>{}
    {\Large \bfseries Everything we do is situated within cultural and historical backdrops of power and oppression.}

    \vspace{1em}

    {\large If we're serious about ethics and justice, we need to be as serious
        about \alert<+>{understanding those histories and power dynamics}
        as we are
        about \alert<+>{imagining potential futures}.}
      % In order to design for justice, we need to understand these contexts.

    \vspace{2em}

    \visible<+->{
      Thinking along these lines forces us to confront how we participate in violent, oppressive systems and practices}\visible<+->{, but that can inspire us.}

    % We design technical and social systems by \alert<-5>{recognizable references}
    % \\\visible<2->{
    % and we can use this to understand systems, and to inform our actions.

    % \visible<3->{
    % \vspace{2em}

    % We can draw from a much deeper well of scholarship than Computer Science to recognize the patterns we're recreating in \alert<-5>{ostensibly} novel technical systems.}
    % }
  \end{center}
\end{frame}



\againframe<7-10>{pastProjects}


\againframe{thesis}

% \againframe<10-12>{pastProjects}


\begin{frame}[standout]
\visible<2>{Finding, translating, and applying theories}
\end{frame}


% \begin{frame}
%     I'll illustrate this by showing that we can draw from the well of historical scholarship to identify the unintentional patterns that we recreate in ostensibly novel technical systems.
% \end{frame}


\begin{frame}<-5>[label=roadmap]{new technologies, old metaphors}

    \begin{itemize}
      \item<1-6,8,10>[] \alert<5>{On-Demand work \hfill Piecework
            \\\hfill{\footnotesize \textsc{\textbf{CHI 2017}}} {\scriptsize(Honorable Mention)}}
      
      \vspace{1em}

      \item<2-4,6-8,10>[] \alert<7>{Artificial Intelligence \hfill Street-level bureaucracies
            \\\hfill{\footnotesize \textsc{\textbf{CHI 2019}}} {\scriptsize(Best Paper)}}

      \vspace{1em}

      \item<3-4,6,8-10>[] \alert<9>{\hfill To Live in Their Utopia
            \\\hfill{\footnotesize \textsc{\textbf{CHI 2021}}}}


    \end{itemize}

\end{frame}

\end{document}
\documentclass[presentation]{subfiles}


\begin{document}


\begin{frame}<-3>[label=theory_evaluation]

\only<-2>{\renewcommand{\baselinestretch}{1.5}
  \begin{tabular}{p{0.9\textwidth}}
{\large\itshape\onslide<-1>{Conceptualization is directed toward the task of generating interpretations of matters already in hand [\dots] But that does not mean that} \onslide<-2>{theory} \onslide<-1>{has only to fit (or, more carefully, to generate cogent interpretations of) realities past; it also} \onslide<-2>{has to survive --- intellectually survive --- realities to come}.}
  \end{tabular}
\renewcommand{\baselinestretch}{1}

  \hfill-- \cite{geertz2008thick}}

\only<3>{
\centering

\vspace{2em}
\alert{Good theories give us analytical power}

for the past

for the present

\alert{for realities to come}}


\end{frame}



\begin{frame}[t]\frametitle{implications}


  A theory of street-level algorithms suggests that, when faced with algorithmic failure, we should reflect on the mechanisms and processes at play with street-level bureaucrats.

% Bureaucratic mechanisms for appeals and justice include:
\begin{itemize}
  \item Ensuring that the person or system reviewing the appeal does not overlap with the person or system who made the initial judgment
  \item Predefined rules for recourse, (e.g. compensating lost income)
  \item Requirements to publish plain-language descriptions of complex systems
\end{itemize}

\end{frame}

\end{document}
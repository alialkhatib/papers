\documentclass[presentation]{subfiles}


\begin{document}
\section{algorithmic bias in justice}

\begin{frame}\frametitle{algorithmic bias in justice}

\begin{columns}
\begin{column}{0.6\textwidth}

Algorithmic systems predict whether defendants are likely to appear at their court date, recommending the level at which to set bail.

\onslide<2>{These algorithms reflect and amplify racial biases in society~(\cite{buolamwini2018gender,lambrecht2018algorithmic,thebault2015avoiding}).}

\end{column}

\begin{column}{0.4\textwidth}
\includegraphics[width=\textwidth]{figures/propublica.png}
\end{column}
\end{columns}

\end{frame}

\begin{frame}
  
  \begin{columns}[t]

  \begin{column}{0.32\textwidth}
    \begin{flushright}
    {\Large \strong{Behavior}}

    \only<1>{Defendant's situation and behavior}
    \only<2->{Defendants from many different jurisdictions, environments, backgrounds}
    % \only<3->{massive amounts of creative information work}
    \end{flushright}
  \end{column}

  \begin{column}{0.36\textwidth}

  \vspace{2em}
  
  \includegraphics[width=\textwidth]{figures/sketches/arrows.pdf}

  \begin{center}
  {\Large \strong{Judgment}}


  \only<1>{Should this defendant be eligible to go free on bail?}

  % \only<2>{Is the factory worker doing the work correctly, or do they need assistance?}


  \end{center}
  \end{column}
  
  \begin{column}{0.32\textwidth}
  \begin{flushleft}
  {\Large\only<1>{\strong{Bureaucrats}}
    \only<2->{\strong{Algorithm}}}

    \only<1>{Judges}

    \only<2->{Algorithmic judges tasked with predicting whether defendants will return for trial}

    % \only<3->{foreman tasked almost exclusively with accepting or rejecting work}
  \end{flushleft}
  \end{column}
  \end{columns}

\only<2->{
\begin{columns}[t]
\begin{column}{0.1\textwidth}
\end{column}
\begin{column}{0.15\textwidth}
\onslide<2>{\strong{Ideally}}


\onslide<3>{\strong{In reality}}
\end{column}

\begin{column}{0.75\textwidth}
\onslide<2>{Account for the circumstances of defendants' environments}


\onslide<3>{Re-enact old cases, even if new
 intersectional ones arise}
\end{column}
\end{columns}
}

\end{frame}

\begin{frame}{takeaways}
  
\alert{Here we have something new}: a street-level bureaucrat interacting with a street-level algorithm. Bureaucrats can resist or buffer the algorithm’s recommendations when needed~(\cite{doi:10.1177/2053951717718855,Veale:2018:FAD:3173574.3174014}).

Even a perfectly fair, transparent, and accountable algorithm will make errors of generalization in cases at the margin. Bureaucrats reason by extension from precedent and case law. How should an algorithm reason?


\end{frame}


\end{document}
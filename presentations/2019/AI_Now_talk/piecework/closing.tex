\documentclass[presentation]{subfiles}


\begin{document}

% \againframe<2->[t]{threads}


% The decentralization and anonymization of on–demand work, especially online crowd work, will continue to make many of its social relationships a struggle. While some workers get to know each other well on forums [104, 54], many never engage in these social spaces. Without intervention, worker relationships and collectivism are likely to be inhibited by this decentralized design. One option is to build worker central- izing points into the platform, for example asking workers to vote on each others’ reputation or allowing groups of workers to collectively reject a task from the platform [157].
% The history of piecework further suggests that relationships be- tween workers and employers might be improved if employers engaged in more human management styles. Instead of dele- gating as many management tasks as possible to an algorithm, it might be possible to build dashboards and other information tools that empower modern crowd work foremen [86]. If the literature on piecework is to be believed, more considerate human management may resolve many of the tensions.
% Reciprocally, crowd work may be able to inform piecework research in this domain. There exists far less literature about piece workers’ relationships than there does today about on– demand workers’ relationships. Two reasons stand out: first, modern platforms are visible to researchers in ways that the sites of piece work labor were not. Second, Anthropology stands on a firmer theoretical and methodological basis than it did at the turn of the 20th century. Malinowski, Boas, Mead,
% and other luminaries throughout the first half of the 20th cen- tury effectively defined Cultural Anthropology as we know it today; participant–observation, the etic and the emic under- standing of culture, and reflexivity didn’t take even a resem- blance of their contemporary forms until these works [103, 19, 108]. On–demand labor today may give us an opportunity to revisit open questions in piecework with a more refined lens.




\begin{frame}{implications}
  

{\renewcommand{\baselinestretch}{1.2}
  \begin{tabular}{p{0.9\textwidth}}
{\large\itshape{As mounting attention increasingly revealed problems in piecework's treatment of workers, workers themselves began to speak out about their frustration with this new regime.}}
  \end{tabular}
\renewcommand{\baselinestretch}{1}}

\end{frame}










% \begin{frame}{Discussion}
% Several goals:
% \begin{itemize}
%   \item Give some historical context to \alert{on--demand work}
%   \item Answer some questions that have been difficult to answer
%   \item Recapture attention toward a valuable sense--making methodology
% \end{itemize}
% \end{frame}

% \begin{frame}{Conclusion}

% \ali{This used to be a big thing about the future of work being dystopian or utopian.
      % Seems sort of coming out of nowhere if I just talk about complexity.}
    % \begin{itemize}
    %   \item What is the \emph{future of work} going to look like?
    % \begin{itemize}
    %   \item Dystopian? Utopian?
    % \end{itemize}
    % \item Probably a little of both
    % % \item
    % \end{itemize}
% \end{frame}

\end{document}
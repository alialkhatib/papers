\documentclass[presentation]{subfiles}


\begin{document}


\begin{frame}[t]\frametitle{concerns}
\only<-2>{Street-level bureaucrats reflect and exercise discretion to support the goals of the institution, but that's resulted in \alert<1>{historically, systematically, marginalized groups}\only<1>{, by way of:}

\only<1>{\begin{itemize}
  \item Systematic brutality committed against people of color, trans people, and other marginalized communities
  \item Criminalization of disempowered groups.
\end{itemize}}
}

\only<2>{
  \centering
  Street-level algorithms are replicating \strong{the same old patterns}
}

\visible<2->{
\centering

% \strong{The same old patterns}

\vfill


\begin{columns}
\begin{column}[T]{0.3\textwidth}
\centering
{\alert<3>{LGBTQ}\\\onslide<2,4>{YouTubers}}
\only<4>{\includegraphics[width=0.8\textwidth]{figures/news/youtubers_sue.png}}
\end{column}
\begin{column}[T]{0.3\textwidth}
\centering
{\alert<3>{Financially precarious} \onslide<2,4>{crowd workers}}
\only<4>{\includegraphics[width=0.8\textwidth]{figures/ghost_work.jpg}}

\end{column}
\begin{column}[T]{0.3\textwidth}
\centering
{\alert<3>{People of color} \onslide<2,4>{in the criminal justice system}}
\only<4>{\includegraphics[width=0.8\textwidth]{figures/race_after_tech.jpg}}
\end{column}
\end{columns}

}





% How might computation amplify the positive aspects of bureaucratic reflexivity and not the negative?

% \vfill


% \visible<3->{What socio-technical configuration combines street-level bureaucratic and street-level algorithmic strengths in the most pro-social way?}
% }

\end{frame}

\end{document}
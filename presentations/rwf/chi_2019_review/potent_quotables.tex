\documentclass[presentation]{subfiles}

\begin{document}
\begin{frame}{Some interesting thoughts}

But technology acts invisibly, often with dubious consent, and typically using a dialect only a few can speak.

The public generally can’t defend themselves against algorithmic bias or seek recourse.


If your algorithmic luck is out, there’s not much you can do except pray.

% Treat everyone the same and you do nothing to address the systemic issues that perpetuate inequality.

\end{frame}
\end{document}
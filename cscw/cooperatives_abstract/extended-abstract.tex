\documentclass{sigchi-ext}
% Please be sure that you have the dependencies (i.e., additional
% LaTeX packages) to compile this example.
\usepackage[T1]{fontenc}
\usepackage{textcomp}
\usepackage[scaled=.92]{helvet} % for proper fonts
\usepackage{graphicx} % for EPS use the graphics package instead
\usepackage{balance}  % for useful for balancing the last columns
\usepackage{booktabs} % for pretty table rules
\usepackage{ccicons}  % for Creative Commons citation icons
\usepackage{ragged2e} % for tighter hyphenation
\usepackage{hyperref}

% Some optional stuff you might like/need.
% \usepackage{marginnote} 
% \usepackage[shortlabels]{enumitem}
% \usepackage{paralist}
% \usepackage[utf8]{inputenc} % for a UTF8 editor only

%% EXAMPLE BEGIN -- HOW TO OVERRIDE THE DEFAULT COPYRIGHT STRIP --
% \copyrightinfo{Permission to make digital or hard copies of all or
% part of this work for personal or classroom use is granted without
% fee provided that copies are not made or distributed for profit or
% commercial advantage and that copies bear this notice and the full
% citation on the first page. Copyrights for components of this work
% owned by others than ACM must be honored. Abstracting with credit is
% permitted. To copy otherwise, or republish, to post on servers or to
% redistribute to lists, requires prior specific permission and/or a
% fee. Request permissions from permissions@acm.org.\\
% {\emph{CHI'14}}, April 26--May 1, 2014, Toronto, Canada. \\
% Copyright \copyright~2014 ACM ISBN/14/04...\$15.00. \\
% DOI string from ACM form confirmation}
%% EXAMPLE END

% Paper metadata (use plain text, for PDF inclusion and later
% re-using, if desired).  Use \emtpyauthor when submitting for review
% so you remain anonymous.
\def\plaintitle{The Potential for Cooperative Piecework Labor Markets}
\def\plainauthor{Ali Alkhatib}
\def\emptyauthor{}
\def\plainkeywords{Qualitative methods; Labor markets}
\def\plaingeneralterms{}

\title{The Potential for Cooperative Piecework Labor Markets}

\numberofauthors{3}
% Notice how author names are alternately typesetted to appear ordered
% in 2-column format; i.e., the first 4 autors on the first column and
% the other 4 auhors on the second column. Actually, it's up to you to
% strictly adhere to this author notation.
\author{%
  \alignauthor{%
    \textbf{Ali Alkhatib}\\
    \affaddr{Stanford University} \\
    \affaddr{Stanford, CA 94305, USA} \\
    \email{ali.alkhatib@cs.stanford.edu} }\alignauthor{%
    \textbf{Sam Witherbee}\\
    \affaddr{Fair Care Labs}\\
    \affaddr{Oakland, CA }\\
    \email{sam@faircarelabs.org} } \vfil \alignauthor{%
    \textbf{Michael Bernstein}\\
    \affaddr{Stanford University}\\
    \affaddr{Stanford, CA 94305, USA} \\
    \email{msb@cs.stanford.edu} }  \vfil \vfil \vfil \vfil \vfil \vfil % \alignauthor{%
    % \textbf{Sixth Author}\\
    % \affaddr{Universit\'e de Auteur-Sud}\\
    % \affaddr{40222 Auteur France}\\
    % \email{author6@author.fr} } \vfil \alignauthor{%
    % \textbf{Third Author}\\
    % \textbf{Fourth Author}\\    
    % \affaddr{L\={e}khaka Interaction Labs}\\
    % \affaddr{Bengaluru 560 080, India}\\
    % \email{author3@anotherco.com} \\
    % \email{author4@hchi.anotherco.com} }\alignauthor{%
    % \textbf{Seventh Author}\\
    % \affaddr{Department of Skrywer}\\
    % \affaddr{University of Umbhali}\\
    % \affaddr{Cape Town, South Africa}\\
    % \email{author7@umbhaliu.ac.za} }
    }

% Make sure hyperref comes last of your loaded packages, to give it a
% fighting chance of not being over-written, since its job is to
% redefine many LaTeX commands.
\definecolor{linkColor}{RGB}{6,125,233}
\hypersetup{%
  pdftitle={\plaintitle},
%  pdfauthor={\plainauthor},
  pdfauthor={\emptyauthor},
  pdfkeywords={\plainkeywords},
  bookmarksnumbered,
  pdfstartview={FitH},
  colorlinks,
  citecolor=black,
  filecolor=black,
  linkcolor=black,
  urlcolor=linkColor,
  breaklinks=true,
}

% \reversemarginpar%

\begin{document}

\maketitle

% Uncomment to disable hyphenation (not recommended)
% https://twitter.com/anjirokhan/status/546046683331973120
\RaggedRight{} 

% Do not change the page size or page settings.
\begin{abstract}
  In this extended abstract, I frame ``gig'' labor and
  information work as similar in nature to piecework of the late 19th century,
  offering historical context to an otherwise seemingly nascent form of labor.
  I further offer potential interventions to affect the frustrations
  workers face as they interact with contemporary systems, especially promoting the
  value of a worker cooperative, and highlighting research framing that work.
  Finally, I discuss very briefly my ongoing work with labor advocacy groups as
  partners.
\end{abstract}

\keywords{\plainkeywords}

\category{H.5.m}{Information interfaces and presentation (e.g.,
  HCI)}{Miscellaneous}\category{See}{\url{http://acm.org/about/class/1998/}}{for
  full list of ACM classifiers. This section is required.}

\section{Introduction}
``Gig'' labor appears to have dramatically reshaped labor markets across industries ranging
driving--for--hire (e.g. Uber, Lyft),
information work (e.g. Amazon Mechanical Turk (AMT), oDesk),
and other industries (e.g. Handy).
Researchers have identified frustration between workers and managers throughout these markets
\cite{Ross,uberAlgorithm}.

Various efforts have been made to ameliorate the challenges workers face,
but fundamentally these interventions have offered only limited respite
\cite{turkopticon,dynamo}.
Workers on AMT
--- ``Turkers'' ---
continue to experience frustration, often precipitated
or at least exacerbated by the very markets upon which they have come to rely.

Such markets are not necessarily novel;
Jacob Riis documents the practice of piecework as a form of labor
as early as the end of the 19th century
\cite{riisOtherSideLives}.
Through his photography,
we begin to see workers paid per piece of clothing completed
--- hence ``piece'' work.
The argument employed at the time
--- that tracking the time that workers are actually working would be challenging,
and that workers prefer to be able to interleave housework with piecework ---
carries surprising parallels today among those who defend Turking and other modern ``gig'' labor.

Thus, I juxtapose Amazon Mechanical Turk \& Uber with the same framing by
thinking about these forms of labor as subclasses of ``piecework'',
well--studied in the late 19th century by Riis through photography
\cite{riisOtherSideLives}.
By contextualizing ``gig'' labor within and perhaps as another name for ``piecework'',
we begin to identify striking similarities between
the workers who stitched strips of denim in their homes, and
their 21st century counterparts who transcribe snippets of audio.

I approached this research ultimately asking how a market
designed, operated, and ultimately owned
by the workers would itself work.
Intuitively, it would dramatically affect the conventional tension between business--owners and workers,
as these groups would become conflated and the interactions become more transparently collaborative rather than adversarial.

\section{Background}
Computational social scientists have documented workers' efforts to circumvent the systems imposed on them by market operators
\cite{uberAlgorithm},
and more directly researchers have observed the continuing effort to resist and critique markets for ``gig" labor, in these cases in the context of online labor, where micro-work on Amazon Mechanical Turk (AMT) predates offline ``gig'' work companies such as Uber
\cite{turkopticon,dynamo}.

Nevertheless, frustrations with these marketplaces persist,
and the trends among emerging marketplaces seem to commoditize workers more and more aggressively.
These ``patches" of existing markets appear to have only marginal effects on the qualities of these markets;
Uber drivers continue to resist the algorithmic matching while walking a fine line to avoid retribution from management, and some of the most frustrating requesters on AMT continue to antagonize Turkers.


% \section{Worker Cooperatives as a Design Intervention}
Worker cooperatives
--- organizations owned by their workers ---
% Grassroots, community-led organizations
are not new;
a substantial body of literature illustrates myriad approaches to guiding communities and assisting in collective action.
Specifically, when we consider the role of insight into collective action, we refer to what Hardin describes as ``directed at an ongoing problem"
\cite{russell1982collective}.
The implication here, he argues, is that the guidance on this form of collective action is dramatically more nuanced than ``one-shot" collective action, demanding an ``anthropological investigation of minute interrelationships".
We might call this ``ongoing" collective action.
Economist Mancur Olson proposes, in part, that collective action depends on some large, generally inactive group in order to succeed
\cite{olsonlogic}.

Hardin posits that collective action is too commonplace for Olson's thesis to hold;
He suggests that the requirements for collective action which Olson theorizes may have changed
--- specifically, lowering the threshold ---
as a result of myriad factors outside of the scope of this research, except to point out that recent work in online collective action prompts further scrutiny of Olson's thesis and the critiques later researchers have levied.
We suggest an alternative consideration: that much of the research in collective action in the space of HCI in fact corroborates the latent community requirement Olson recommends.
Myriad collective action endeavors seem to succeed in part because they precipitate a collective of latent, willing participants in some form of community action
\cite{catalyst,dynamo,foundry}.
% Cooperatively owned organizations are not novel;
% case studies \cite{craig1992behavior,estrin1987productivity} and
% theoretical grounding \cite{mellor1988worker}
% substantively inform the study of cooperatives and economics.
% Yet more research explores the complicated nature of coordinating collective action
% \cite{polletta2012freedom,russell1982collective,olsonlogic}.
% Hardin in particular differentiates between specific instances of collective action,
% calling for a sort of economic game theory to understand,
% as compared to the practice of ongoing and sustained collective action,
% which he says calls for an ``anthropological investigation'' to appreciate the
% many nuanced ongoing relationships which make this latter phenomenon possible
% \cite{russell1982collective}.

% The findings of research in offline communities do not necessarily map directly to online communities, 
% simply by virtue of those communities being predominantly digitally mediated and therefore facilitating both quantitatively and qualitatively different interactions
% \cite{miller2011understanding}.
% Dynamo,
% previously mentioned,
% identified various ways to facilitate collective action in an online community
% \cite{dynamo}.
% Similarly, the unique features of the Internet both
% problematize some of what we know about
% the ongoing collective action represented by cooperative labor groups
% and
% invite us to explore novel ways of enabling and facilitating collaborative market organization.

A robust and growing body of research exploring collective action and movements enabled by the Internet adds to a body of knowledge previously uninformed by the tools the Internet affords.
Where collective action research coordinated and executed offline describes challenges symptomatic of social structures,
researchers of online communities can and do offer design guidance for the structure of online communities
\cite{Hirsch:2009:FLA:1516016.1516024}.
Substantial contributions deeply investigating online communities, such as Wikipedia, have lent system designers guidance in designing communities geared toward some ongoing collective action online
\cite{Nov:2007:MW:1297797.1297798,wikipediansBornNotMade,whyWikipedians}.
In this last case, studies of Wikipedia and its users
--- known colloquially as ``Wikipedians" ---
is especially instructive, as it addresses the distinction Olson makes regarding ``one-shot" collective action and what we will call ``ongoing" collective action.

\section{Ongoing Work}
In collaboration with
the National Domestic Workers Alliance (NDWA)
and their innovation arm
the Fair Care Labs,
I am continuing the work we started at
the FUSE Labs at Microsoft Research in Redmond
of exploring cooperative labor markets.
% designing in collaboration with workers.
The immediate hope is to produce a system
that workers can adopt with minimal technical expertise,
allowing them to offer freelance work on their own.

We're hoping to precipitate a community of workers with
mutual interest in success and a willingness to cooperate together,
allowing the formation of a loose cooperative of workers who collectively
invest in community needs, share risks, and bargain for benefits like insurance.

My hope is that this work can open the door for a ``labor protocol'' ---
a standard enabling workers to run their own systems provided they conform to specifications
describing how they should communicate with other systems.
A standard protocol
--- especially one which allows for the transferral of data ---
will afford workers the ability to transfer their
reputations,
benefits,
and ultimately their work to more preferable markets.

% This work is informed by months of fieldwork
% and iterative design in collaboration with workers in Washington and California.

% Specifically, I'm working with
% the Fair Care Labs
% to design a system that makes it easier for workers to schedule and manage their work.


% In the future,
% we hope to take this system and use it as a starting point to allow workers
% to coordinate, determining norms and policies,
% as well as deciding collectively whether and for what to invest in as a group.






% \section{ACM Copyrights \& Permission}
% Accepted extended abstracts and papers will be distributed in the
% Conference Publications. They will also be placed in the ACM Digital
% Library, where they will remain accessible to thousands of researchers
% and practitioners worldwide. To view the ACM's copyright and
% permissions policy, see:
% \url{http://www.acm.org/publications/policies/copyright_policy}.

% \marginpar{%
%   \vspace{-45pt} \fbox{%
%     \begin{minipage}{0.925\marginparwidth}
%       \textbf{Good Utilization of the Side Bar} \\
%       \vspace{1pc} \textbf{Preparation:} Do not change the margin
%       dimensions and do not flow the margin text to the
%       next page. \\
%       \vspace{1pc} \textbf{Materials:} The margin box must not intrude
%       or overflow into the header or the footer, or the gutter space
%       between the margin paragraph and the main left column. The text
%       in this text box should remain the same size as the body
%       text. Use the \texttt{{\textbackslash}vspace{}} command to set
%       the margin
%       note's position. \\
%       \vspace{1pc} \textbf{Images \& Figures:} Practically anything
%       can be put in the margin if it fits. Use the
%       \texttt{{\textbackslash}marginparwidth} constant to set the
%       width of the figure, table, minipage, or whatever you are trying
%       to fit in this skinny space.
%     \end{minipage}}\label{sec:sidebar} }

% \section{Page Size}
% All SIGCHI submissions should be US letter (8.5 $\times$ 11
% inches). US Letter is the standard option used by this \LaTeX\
% template.

% \section{Text Formatting}
% Please use an 8.5-point Verdana font, or other sans serifs font as
% close as possible in appearance to Verdana in which these guidelines
% have been set. Arial 9-point font is a reasonable substitute for
% Verdana as it has a similar x-height. Please use serif or
% non-proportional fonts only for special purposes, such as
% distinguishing \texttt{source code} text.

% \subsubsection{Text styles}
% The \LaTeX\ template facilitates text formatting for normal (for body
% text); heading 1, heading 2, heading 3; bullet list; numbered list;
% caption; annotation (for notes in the narrow left margin); and
% references (for bibliographic entries). Additionally, here is an
% example of footnoted\footnote{Use footnotes sparingly, if at all.}
% text. As stated in the footnote, footnotes should rarely be used.

% \begin{figure}
%   \includegraphics[width=0.9\columnwidth]{figures/sigchi-logo}
%   \caption{Insert a caption below each figure.}~\label{fig:sample}
% \end{figure}

% \subsection{Language, style, and content}
% The written and spoken language of SIGCHI is English. Spelling and
% punctuation may use any dialect of English (e.g., British, Canadian,
% US, etc.) provided this is done consistently. Hyphenation is
% optional. To ensure suitability for an international audience, please
% pay attention to the following:

% \begin{table}
%   \centering
%   \begin{tabular}{l r r r}
%     % \toprule
%     & & \multicolumn{2}{c}{\small{\textbf{Test Conditions}}} \\
%     \cmidrule(r){3-4}
%     {\small\textit{Name}}
%     & {\small \textit{First}}
%       & {\small \textit{Second}}
%     & {\small \textit{Final}} \\
%     \midrule
%     Marsden & 223.0 & 44 & 432,321 \\
%     Nass & 22.2 & 16 & 234,333 \\
%     Borriello & 22.9 & 11 & 93,123 \\
%     Karat & 34.9 & 2200 & 103,322 \\
%     % \bottomrule
%   \end{tabular}
%   \caption{Table captions should be placed below the table. We
%     recommend table lines be 1 point, 25\% black. Minimize use of
%     table grid lines.}~\label{tab:table1}
% \end{table}

% \begin{itemize}\compresslist%
% \item Write in a straightforward style. Use simple sentence
%   structure. Try to avoid long sentences and complex sentence
%   structures. Use semicolons carefully.
% \item Use common and basic vocabulary (e.g., use the word ``unusual''
%   rather than the word ``arcane'').
% \item Briefly define or explain all technical terms. The terminology
%   common to your practice/discipline may be different in other design
%   practices/disciplines.
% \item Spell out all acronyms the first time they are used in your
%   text. For example, ``World Wide Web (WWW)''.
% \item Explain local references (e.g., not everyone knows all city
%   names in a particular country).
% \item Explain ``insider'' comments. Ensure that your whole audience
%   understands any reference whose meaning you do not describe (e.g.,
%   do not assume that everyone has used a Macintosh or a particular
%   application).
% \item Explain colloquial language and puns. Understanding phrases like
%   ``red herring'' requires a cultural knowledge of English. Humor and
%   irony are difficult to translate.
% \item Use unambiguous forms for culturally localized concepts, such as
%   times, dates, currencies, and numbers (e.g., ``1-5- 97'' or
%   ``5/1/97'' may mean 5 January or 1 May, and ``seven o'clock'' may
%   mean 7:00 am or 19:00). For currencies, indicate equivalences:
%   ``Participants were paid {\fontfamily{txr}\selectfont \textwon}
%   25,000, or roughly US \$22.''
% \item Be careful with the use of gender-specific pronouns (he, she)
%   and other gender-specific words (chairman, manpower,
%   man-months). Use inclusive language (e.g., she or he, they, chair,
%   staff, staff-hours, person-years) that is gender-neutral. If
%   necessary, you may be able to use ``he'' and ``she'' in alternating
%   sentences, so that the two genders occur equally
%   % often~\cite{Schwartz:1995:GBF}.
% \item If possible, use the full (extended) alphabetic character set
%   for names of persons, institutions, and places (e.g.,
%   Gr{\o}nb{\ae}k, Lafreni\'ere, S\'anchez, Nguy{\~{\^{e}}}n,
%   Universit{\"a}t, Wei{\ss}enbach, Z{\"u}llighoven, \r{A}rhus, etc.).
%   These characters are already included in most versions and variants
%   of Times, Helvetica, and Arial fonts.
% \end{itemize}

% % \begin{figure}
% %   \includegraphics[width=.9\columnwidth]{figures/ea-figure2}
% %   \caption{If your figure has a light background, you can set its
% %     outline to light gray, like this, to make a box around
% %     it.}\label{fig:bats}
% % \end{figure}

% \begin{marginfigure}[-35pc]
%   \begin{minipage}{\marginparwidth}
%     \centering
%     \includegraphics[width=0.9\marginparwidth]{figures/cats}
%     \caption{In this image, the cats are tessellated within a square
%       frame. Images should also have captions and be within the
%       boundaries of the sidebar on page~\pageref{sec:sidebar}. Photo:
%       \cczero~jofish on Flickr.}~\label{fig:marginfig}
%   \end{minipage}
% \end{marginfigure}

% \section{Figures}
% The examples on this and following pages should help you get a feel
% for how screen-shots and other figures should be placed in the
% template. Your document may use color figures (see
% Figures~\ref{fig:sample}), which are included in the page limit; the
% figures must be usable when printed in black and white. You can use
% the \texttt{\marginpar} command to insert figures in the (left) margin
% of the document (see Figure~\ref{fig:marginfig}). Finally, be sure to
% make images large enough so the important details are legible and
% clear (see Figure~\ref{fig:cats}).

% \section{Tables}
% You man use tables inline with the text (see Table~\ref{tab:table1})
% or within the margin as shown in Table~\ref{tab:table2}. Try to
% minimize the use of lines (especially vertical lines). \LaTeX\ will
% set the table font and captions sizes correctly; the latter must
% remain unchanged.

% \section{Accessibility}
% The Executive Council of SIGCHI has committed to making SIGCHI
% conferences more inclusive for researchers, practitioners, and
% educators with disabilities. As a part of this goal, the all authors
% are asked to work on improving the accessibility of their
% submissions. Specifically, we encourage authors to carry out the
% following five steps:
% \begin{itemize}\compresslist%
% \item Add alternative text to all figures
% \item Mark table headings
% \item Generate a tagged PDF
% \item Verify the default language
% \item Set the tab order to ``Use Document Structure''
% \end{itemize}

% For links to instructions and resources, please see:
% \url{http://chi2016.acm.org/accessibility}

% Unfortunately good tools do not yet exist to create tagged PDF files
% from Latex. \LaTeX\ users will need to carry out all of the above
% steps in the PDF directly using Adobe Acrobat, after the PDF has been
% generated.

% For more information and links to instructions and resources, please
% see:
% \url{http://chi2016.acm.org/accessibility}.

% \begin{figure*}
%   \centering
%   \includegraphics[width=1.4\columnwidth]{figures/map}
%   \caption{In this image, the map maximizes use of space. You can make
%     figures as wide as you need, up to a maximum of the full width of
%     both columns. Note that \LaTeX\ tends to render large figures on a
%     dedicated page. Image: \ccbynd~ayman on Flickr.}~\label{fig:cats}
% \end{figure*}

% \section{Producing and Testing PDF Files}
% We recommend that you produce a PDF version of your submission well
% before the final deadline. Your PDF file must be ACM DL Compliant and
% meet stated requirements,
% \url{http://www.sheridanprinting.com/sigchi/ACM-SIG-distilling-settings.htm}.

% \marginpar{\vspace{-23pc}So long as you don't type outside the right
%   margin or bleed into the gutter, it's okay to put annotations over
%   here on the left, too; this annotation is near Hawaii. You'll have
%   to manually align the margin paragraphs to your \LaTeX\ floats using
%   the \texttt{{\textbackslash}vspace{}} command.}

% \begin{margintable}[1pc]
%   \begin{minipage}{\marginparwidth}
%     \centering
%     \begin{tabular}{r r l}
%       & {\small \textbf{First}}
%       & {\small \textbf{Location}} \\
%       \toprule
%       Child & 22.5 & Melbourne \\
%       Adult & 22.0 & Bogot\'a \\
%       \midrule
%       Gene & 22.0 & Palo Alto \\
%       John & 34.5 & Minneapolis \\
%       \bottomrule
%     \end{tabular}
%     \caption{A simple narrow table in the left margin
%       space.}~\label{tab:table2}
%   \end{minipage}
% \end{margintable}
% Test your PDF file by viewing or printing it with the same software we
% will use when we receive it, Adobe Acrobat Reader Version 10. This is
% widely available at no cost. Note that most
% reviewers will use a North American/European version of Acrobat
% reader, so please check your PDF accordingly.

% \section{Acknowledgements}
% We thank all the volunteers, publications support, staff, and authors
% who wrote and provided helpful comments on previous versions of this
% document. As well authors 1, 2, and 3 gratefully acknowledge the grant
% from NSF (\#1234--2222--ABC). Author 4 for example may want to
% acknowledge a supervisor/manager from their original employer. This
% whole paragraph is just for example. Some of the references cited in
% this paper are included for illustrative purposes only.

% \section{References Format}
% Your references should be published materials accessible to the
% public. Internal technical reports may be cited only if they are
% easily accessible and may be obtained by any reader for a nominal
% fee. Proprietary information may not be cited. Private communications
% should be acknowledged in the main text, not referenced (e.g.,
% [Golovchinsky, personal communication]). References must be the same
% font size as other body text. References should be in alphabetical
% order by last name of first author. Use a numbered list of references
% at the end of the article, ordered alphabetically by last name of
% first author, and referenced by numbers in brackets. For papers from
% conference proceedings, include the title of the paper and the name of
% the conference. Do not include the location of the conference or the
% exact date; do include the page numbers if available. 

% References should be in ACM citation format:
% \url{http://www.acm.org/publications/submissions/latex_style}.  This
% includes citations to Internet
% resources~\cite{CHINOSAUR:venue,cavender:writing,psy:gangnam}
% according to ACM format, although it is often appropriate to include
% URLs directly in the text, as above. Example reference formatting for
% individual journal articles~\cite{ethics}, articles in conference
% proceedings~\cite{Klemmer:2002:WSC:503376.503378},
% books~\cite{Schwartz:1995:GBF}, theses~\cite{sutherland:sketchpad},
% book chapters~\cite{winner:politics}, an entire journal
% issue~\cite{kaye:puc},
% websites~\cite{acm_categories,cavender:writing},
% tweets~\cite{CHINOSAUR:venue}, patents~\cite{heilig:sensorama}, and
% online videos~\cite{psy:gangnam} is given here.  See the examples of
% citations at the end of this document and in the accompanying
% \texttt{BibTeX} document. This formatting is a edited version of the
% format automatically generated by the ACM Digital Library
% (\url{http://dl.acm.org}) as ``ACM Ref''. DOI and/or URL links are
% optional but encouraged as are full first names. Note that the
% Hyperlink style used throughout this document uses blue links;
% however, URLs in the references section may optionally appear in
% black.


\balance{}
\bibliographystyle{SIGCHI-Reference-Format}
\bibliography{references}

\end{document}

%%% Local Variables:
%%% mode: latex
%%% TeX-master: t
%%% End:

\documentclass[20pt,margin=0in,innermargin=0in]{tikzposter}
\geometry{paperwidth=36in,paperheight=24in}
\usepackage[utf8]{inputenc}
\usepackage{amsmath}
\usepackage{amsfonts,sourceserifpro}
\usepackage{amsthm}
\usepackage{amssymb}
\usepackage{mathrsfs,blindtext,nth,microtype}
\usepackage{graphicx}
\usepackage{adjustbox}
\usepackage{enumitem}
\usepackage[backend=bibtex,style=numeric]{biblatex}
\usepackage{stanford-theme}

\usepackage{mwe} % for placeholder images
\bibliography{references}


% set theme parameters
\tikzposterlatexaffectionproofoff
\usetheme{StanfordTheme}
\usecolorstyle{StanfordStyle}


\title{Poster Template}
\author{person name\\\texttt{personname@stanford.edu}}
\institute{Department of Computer Science, Stanford University} % you probably don't need this at a Stanford event!
\titlegraphic{\includegraphics[height=0.125\textheight]{stanford}}


\begin{document}

\maketitle[titletotopverticalspace=1in]
\centering

\begin{columns}
    \column{0.3}
    \block{\nth{1} Column}{

    You can have as many columns as you want, and you can make them whatever widths you want.
    You can even have columns within columns, but I'm not showing it here because it would've looked weird.
    But if that makes sense for your purposes, you can do that.

    \vspace{1em}

    You should take this document as a starting point, if you take it at all.
    It's no good if everyone copies this layout, pastes in their text, and goes off to the printers.
    It's totally worth thinking for a minute or two about if you want a 2/3-1/3 layout, or whatever

    % \begin{subcolumns}
    %   \subcolumn{.6}
    %   \block{test}{something}
    %   \subcolumn{.4}
    %   \block{test2}{hi}
    % \end{subcolumns}

% \blindtext[1]
    }
    
    \block{\nth{2} Column}{
    \blindtext[1]

    \vspace{1em}

    {\scshape Don't get too close to margins.

    No, seriously. There's nothing but trouble here.}

    }

    \column{0.45}
    \block{\nth{3} Column}{
        You can have some formulas I guess:

        \begin{align*} q^{-3} & \le \frac{\overline{\sqrt{2}-\emptyset}}{\tilde{\omega} \left( e, \dots, \frac{1}{P ( A )} \right)} \wedge p \left( \bar{K}^{-5}, \tilde{m} \right) \\ & = \max_{B \to \emptyset}  1 \pm \dots \cup \pi \left(-q ( d ), \dots, \mathscr{{C}}'' \right)  \\ & \le \left\{ 1^{-7} \colon \cosh^{-1} \left(-\kappa \right) \le \max \int_{\hat{M}} \tanh \left( C^{5} \right) \,d \theta \right\} \\ & \le \prod  \cosh^{-1} \left( \pi^{-8} \right) + \dots \vee \omega \left(-\pi, \infty \sqrt{2} \right) \end{align*}

        You can also have a figure:
        
        \vspace{1em}
        \begin{tikzfigure}[Here's where captions go. Captions are good.]
            \includegraphics[width=0.8\linewidth]{example-image}
        \end{tikzfigure}
        % \vspace{1em}
    }




    \column{0.25}
    \block{\nth{4} Column}{
    % \coloredbox[bgcolor=CardinalRed,fgcolor=White,framecolor=White]{
    I think you're getting the idea.

    % But you can also do a colored box of text.
    % }
        
    \vspace{6em}
    }
    

    \block{Acknowledgments}{
    Be sure to thank people who made the work you did possible --- specifically, funding sources, people and organizations who provided data, etc\dots

    \vspace{2em}

    There's no need to acknowledge your collaborators (the people you list at the top of the poster), because their contribution is being acknowledged in the form of co-authorship.

    \vspace{2em}

    \textit{Fun fact: author order is ``a whole thing''; you should check with your mentor about the norms in whatever field you're publishing to.}
    }
    
    % \block{References}{
    %     \vspace{-1em}
    %     \begin{footnotesize}
    %     \printbibliography[heading=none]
    %     You probably won't need to list all the references, but if you do, you can uncomment this block
    %     \end{footnotesize}
    % }
\end{columns}
\end{document}